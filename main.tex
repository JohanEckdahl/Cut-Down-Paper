%------------------------------------------------------------------------------
% Template file for the submission of papers to IUCr journals in LaTeX2e
% using the iucr document class
% Copyright 1999-2013 International Union of Crystallography
% Version 1.6 (28 March 2013)
%------------------------------------------------------------------------------


\documentclass[preprint]{iucr}              % DO NOT DELETE THIS LINE
%\documentclass{iucr}

     %-------------------------------------------------------------------------
     % Information about journal to which submitted
     %-------------------------------------------------------------------------
     \journalcode{S}              % Indicate the journal to which submitted
                                  %   A - Acta Crystallographica Section A
                                  %   B - Acta Crystallographica Section B
                                  %   C - Acta Crystallographica Section C
                                  %   D - Acta Crystallographica Section D
                                  %   E - Acta Crystallographica Section E
                                  %   F - Acta Crystallographica Section F
                                  %   J - Journal of Applied Crystallography
                                  %   M - IUCrJ
                                  %   S - Journal of Synchrotron Radiation

\begin{document}                  % DO NOT DELETE THIS LINE

     %-------------------------------------------------------------------------
     % The introductory (header) part of the paper
     %-------------------------------------------------------------------------

     % The title of the paper. Use \shorttitle to indicate an abbreviated title
     % for use in running heads (you will need to uncomment it).

\title{Effects of Temperature Gradients in Crystal Monochromators}
%\shorttitle{Short Title}

     % Authors' names and addresses. Use \cauthor for the main (contact) author.
     % Use \author for all other authors. Use \aff for authors' affiliations.
     % Use lower-case letters in square brackets to link authors to their
     % affiliations; if there is only one affiliation address, remove the [a].

\author[a]{Johan}{Eckdahl}
\cauthor[a]{Peter}{Sondhauss}{peter.sondhauss@maxiv.lu.se}{address if different from \aff}

\aff[a]{MAX IV Laboratory \country{Sweden}}


     % Use \shortauthor to indicate an abbreviated author list for use in
     % running heads (you will need to uncomment it).

%\shortauthor{Soape, Author and Doe}

     % Use \vita if required to give biographical details (for authors of
     % invited review papers only). Uncomment it.

%\vita{Author's biography}

     % Keywords (required for Journal of Synchrotron Radiation only)
     % Use the \keyword macro for each word or phrase, e.g. 
     % \keyword{X-ray diffraction}\keyword{muscle}

%\keyword{keyword}

     % PDB and NDB reference codes for structures referenced in the article and
     % deposited with the Protein Data Bank and Nucleic Acids Database (Acta
     % Crystallographica Section (d). Repeat for each separate structure e.g
     % \PDBref[dethiobiotin synthetase]{1byi} \NDBref[d(G$_4$CGC$_4$)]{ad0002}

%\PDBref[optional name]{refcode}
%\NDBref[optional name]{refcode}

\maketitle                        % DO NOT DELETE THIS LINE


\begin{synopsis}
Supply a synopsis of the paper for inclusion in the Table of Contents.
\end{synopsis}

\begin{abstract}
Crystal monochromators are standard in modern synchrotron hard X-ray beamlines. Their role is reducing a polychromatic photon beam to a small range of wavelengths, typically within a bandwidth of 0.01-0.1\%. These monochromators work through the principle of Bragg diffraction where crystal lattice spacing, photon energy and incident angle define a narrow range of high reflectivity. The crystal surface where the polychromatic beam hits first receives an enormous heat load as it absorbs the vast majority of the beam power, which is in the kilowatt regime. The heat creates thermoelastic stress that warps the crystal surface and alters its lattice spacing. This paper investigates the effects of this on the reflected beam through the help of simulations. Calculations using COMSOL for finite element analysis of heat transport, thermal expansion, and elastic strain, and SHADOW3 for ray tracing of radiation transfer were coordinated by a framework called MASH. First, a numerical study of diffraction in Si~111 and Si~333 with a uniform strain depth gradient was performed. Then hard X-ray sources of varying energies were used to study reflectivity in Si~111 and Si~333 for the cases of strained and unstrained crystals. Monochromator performance, i.e. the attenuation and bandwidth of the beam after monochromation, was recorded. Significance of the results is discussed and further study proposed.


\end{abstract}


     %-------------------------------------------------------------------------
     % The main body of the paper
     %-------------------------------------------------------------------------
     % Now enter the text of the document in multiple \section's, \subsection's
     % and \subsubsection's as required.

\section{Introduction}

The performance of synchrotron radiation facilities is ever increasing. Constant demand for improvement pushes scientists and engineers to look at every aspect of accelerator and beamline design. Resulting new designs, methods, and technologies provide for enormous boosts to the brilliance of X-ray beams, which itself poses its own challenges.

One of the most impacted components of this trend are the beamline crystal monochromators. These devices typically absorb 99.9\% of incident beam power~\cite{willmott} and are generally placed directly after the front end. This results in an enormous heat load and significant distortion of the crystal. Among the potential effects of this are losses in flux and a change in bandwidth.

Computer simulations of monochromator throughput are commonplace but have had a history of not making entirely accurate predictions. Reference~\cite{innacuratepredictions} reviews this issue in a time where water-cooled crystals were commonplace. Despite the advent of modern liquid nitrogen cooling the situation has only been aggravated due to the advent of higher output X-ray sources with smaller source size and lower source emittance. In order for beamline scientists to better understand their monochromators, and thereby more effectively compensate for performance losses, aspects such as lattice strain should be included in computer simulations. Also, this understanding may aid in the design process of new beamlines and therefore continuing the trend of higher performing synchrotron facilities.

Previous work on this topic has had interesting results but is believed to have not given a well-rounded picture. In particular, a significant work performed by Zhang et al~\cite{Zhang} used a free parameter to fit prediction to experimental results. In this paper the authors studied crystal surface deformation, rocking curve width and photon throughput. Though many detailed methods were used the predictions were fit to experiments by tuning the heat transfer coefficient between the cooling blocks and silicon crystal.

The studies described here aim to expand and improve previous work on the subject and thoroughly demonstrate the significance of the effect of lattice strain in crystal monochromators. Of particular merit, this study provides detailed ray tracing simulations over a large variety of beam conditions which inspire an intuition of the effects of heat load on a monochromator.

This investigation is motivated by the newly built MAX IV synchrotron facility in Lund, Sweden with its small beam size and resulting high intensity on the crystals. The planned beamline ForMAX~\cite{formax} serves as the first subject of these new methods.

\section{Ray Tracing and Finite Element Analysis}

The bulk of the study is performed using combined ray tracing and finite element analysis. A highly automated framework called MASH~\cite{mash} runs simulations combining the two. These allow for comprehensive statistics and the capability of running many detailed simulations with minimal setup and user interaction.

Ray tracing is accomplished using SHADOW3~\cite{shadow3}, a popular, well-tested and powerful Fortran code under development since the 1970s. SHADOW3 simulates the generation of X-rays in the insertion device and their transfer through optical components. Monochromator reflectivities are calculated using the dynamical theory of diffraction \cite{dynamicaltheory}, \cite{asymmetricdiffraction}.

Finite element analysis is performed using COMSOL Multiphysics~\cite{comsol}. This program receives a power profile of the absorbed X-rays from SHADOW3 and uses it to calculate heating, mechanical stress and strain in the lattice of the first monochromator. These values are incorporated in subsequent ray tracing.

This intermingling of SHADOW3 and COMSOL is made possible by MASH. Usage of MASH begins with a web interface where one defines properties of the source, beamline optics and sampling methods as well as so-called "parameter sweeps" where basically any beamline parameter can be scanned through automatically. MASH then delegates SHADOW3 and COMSOL using these parameters and supervises their interaction.

\section{Models}

\subsection{Crystal with a Uniform Strain Depth Gradient}\label{strain_gradient} 
As described by Chukhovskii~\cite{Chukhovskii} and Taupin~\cite{Taupin}, a strain gradient perpendicular to a crystal surface can have a dramatic effect on reflectivity. An analysis was made in order to deem the importance of such a strain gradient for a crystal under heat load. If insignificant, strain can be modeled as a 2-dimensional field of varying interlattice spacing and complexity of the simulations would be drastically reduced as major modifications to the existing SHADOW3 ray tracing code could be avoided. Computer code for calculating crystal reflectivity within the framework of the dynamical theory of X-ray diffraction was written, validated and provided by Reference~\cite{coins}.


\subsection{FEA Models}

Two geometric models for FEA study in COMSOL were built. The first model was simply a bare block of silicon with a defined constant thermal resistance on two opposite sides. The second is a model of the monochromator found at the MAX IV beamlines NanoMAX and BioMAX including copper cooling blocks flushed with liquid nitrogen. Parameters used in simulations are found in Appendix~\ref{feaparameters}.

\subsubsection{Bare Silicon Block Model}
This model is rudimentary but its simplicity and low computational costs beneficial for quick testing. Another advantage is that it is very general as few assumptions are made about design parameters. It consists of a $20\times 40\times 50~$mm$^3$ block of pure silicon crystal with beam incidence on the top surface and set thermal resistance on two of its sides. The remaining surfaces are thermally insulated. The crystal is kinematically mounted in three points wit ha minimum of motional restrictions meaning it is fixed in place without limiting the ability to expand or contract in all three directions. The model also features a section of exceptionally fine mesh where the beam is incident. It is fine enough to outline the entire beam profile and has a gradual increase in size until reaching the normal mesh size of the rest of the silicon body. This fine mesh is important in order to accurately sample the power footprint and provide suitably accurate predictions of strain distribution. The meshed silicon block and close-up image of the mesh adjustment are seen in Figure~\ref{fig:bare_silicon}.

COMSOL does not provide a temperature dependent template for silicon in its materials library. Therefore one must derive and insert functions of temperature for the secant coefficient of thermal expansion, $\alpha$, and the thermal conductivity, $\kappa$. This was done using techniques described in Reference~\cite{mash}.

\subsubsection{NanoMAX and BioMAX Models}\label{feamodels}
The second model is a simplified replica of a monochromator found in the beamline optics of NanoMAX and BioMAX at MAX IV and is seen in Figure~\ref{fig:nanomaxcomsol}. The monochromators were built by the same manufacturer and for the purposes of this simulation can be considered identical. Between each cooling block and the crystal an indium foil of 50~$\mu$m was added. A measured heat transfer coefficient for the NanoMAX monochromator of 7000~W/K/m$^{2}$ was used for this interface. Turbulent fluid flow was added to the model to get a realistic idea of the thermal resistance at the pipe periphery.

The model used for simulating the fluid flow was the $k-\epsilon$ RANS model. By adjusting the fluid flow rate a suitable speed is found that prevents the liquid nitrogen from heating by more than one kelvin. This allows the neglection of any temperature dependence in the properties of nitrogen.

Stress and strain induced by mechanical mounting while the crystal expanded or shrunk under the heat load was also studied in COMSOL. Even under extreme conditions these strains were not seen near beam incidence on the crystal. However, the mounting of the real monochromator is complex and the mechanical stress difficult to predict. This may be considered in the future but for this study is ignored.

\section{Simulations}

\subsection{Artificial Gaussian Source}
In order to understand the fundamental relation of the monochromator performance with certain source parameters, a generic light source was used where one has control over each parameter separately. Due to its increased sensitivity to strain, a higher order reflection, here Si~333, was studied. The simplest model was a good starting point. The bare silicon block was set a distance of 20~m from the artificial Gaussian X-ray source. The X-ray source had a flat spectrum, i.e. spectral power flux, ranging from 2 to 40~keV. The spatial distributions in transverse and longitudinal spatial directions as well as the angular distributions were represented by Gaussian functions with parameters given in Table~\ref{gaussian_table}. Unless otherwise stated the total power of the beam was set to 150~W. So-called ``continuation planes", i.e. virtual planes used for probing beam parameters, were set up after each crystal. Optical elements and their distances can be seen in Figure~\ref{fig:dcmtracing}.

\subsection{Undulator and Wiggler Sources}\label{undulatorsource}
More simulations were performed using an undulator and a wiggler source incident on the bare silicon crystal model. The undulator is modeled after the type used at both BioMAX and NanoMAX and its parameters are given in Table~\ref{ivubiomax}. These simulations were also performed over the range of 6-35~keV. The insertion device gap was programatically adjusted to provide the highest on axis flux at the given photon energy. The optical setup was otherwise the same as for the Gaussian source simulations.

The wiggler and corresponding optics match what is found in the MAX IV beamline Balder. The wiggler source radiation cone is limited by front end apertures, reflects from a vertically deflecting collimating mirror and finally passes a vertically deflecting double crystal monochromator. Wiggler parameters are given in Table~\ref{ivwbalder}.

\subsection{Second Crystal Compensation}

An argument can be made that attenuation due to strain may be compensated for by adjusting the angle of the second crystal relative to the first. For the most intense part of the beam strain in the first crystal will be highest and therefore also the difference between the lattice spacing of the two crystals. One could adjust the second crystal to allow for greater reflection of this intense portion at the sacrifice of less intense parts. Although, this optimization has to be repeated every time the monochromator is tuned to a different photon energy. This is certainly feasible at beamlines working at more or less fixed photon energies. For a fast scanning monochromator this is most likely impractical. Results of this tweaking in simulation is provided.

\section{Results and Discussion}

The first result is concerning the effects of a uniform strain gradient in Si~111 and Si~333 as described in Section~\ref{strain_results}. Secondly, the crystal surface deformation is analyzed in Section~\ref{deformation}. In Section~\ref{parameterscans} parameter scans are described using an artificial Gaussian source and including the effects of a two dimensional strain field across the surface of the first crystal (without strain variation into the crystal bulk). Further studies with undulator and wiggler sources are described in Section~\ref{undulatorwiggler}.

\subsection{Effects of Uniform Strain Depth Gradient in Si~111 and Si~333}\label{strain_results}

To determine the strain depth gradient present in a crystal under typical operating conditions the simple bare silicon model was used in COMSOL. An undulator source produced by SHADOW3 of 130~W total power and footprint of approximately 1$\times$2~mm$^2$ was incident on the finely meshed area of the crystal surface. Side cooling temperatures were set at intervals with each one producing a solution for the extremum of temperature and strain depth gradient found in the crystal. The strain depth gradient is a result of temperature dependent thermal expansion and temperature gradients throughout the crystal body. The numerical methods described in Section~\ref{strain_gradient} were used with photon energies of 10 and 20~keV to generate the plots in Figure~\ref{fig:111USG} showing Si~111 reflection curves. From the plots one can notice that effects of the strain gradient do not clearly develop until around a maximum temperature of 168~K, which is exceptionally warm. Even at this temperature the effect on the rocking curve is negligible as the oscillations in the distorted curve still outline the undistorted curve closely. With this data it is concluded that the effects of a strain depth gradient on the Si~111 reflection need not be considered for the given source within the operating temperatures and photon energies studied in this paper.

However,  for higher order reflections, say Si~333, conditions need to be considered more carefully. The same methods were used to generate plots in Figure~\ref{fig:333USG}. Note that even around an operating temperature of 133~K the effects are clearly visible. However, when averaging out these oscillations the curve is not considerably deviant. Reaching higher temperatures the effect is harder to ignore. At 168~K the curve has drifted and is considerably deviant. Unfortunately, the effect of a strain depth gradient in SHADOW3 simulations would be difficult to incorporate and laborious to maintain if it would not become part of SHADOW3's main development branch. However, a moderately powered undulator, e.g. 100~W, and even the wiggler studied in this paper maintain a maximum temperature near or slightly below 135~K. Therefore the effects will be ignored. However, due to its significance, future efforts to incorporate the phenomenon into simulations is likely. For an interesting look at monochromator throughput studies accounting for uniform strain gradients see Reference~\cite{mocellaUSG}.

\subsection{Crystal Surface Deformation}\label{deformation}

Deformation of the crystal affects reflection by changes in the incident angle. Conveniently, SHADOW3 considers the effects of surface deformation. Therefore, manual analysis is only necessary in order for the reader to understand qualitatively the effect of deformation on the output.

Briefly, surface distortion calculated in the bare silicon model is presented. One can see in Figure~\ref{fig:ydeformation} charts of the \textit{z} displacement as a function of \textit{y} (meridional direction). The position of \textit{x} is set to 0 so that the data is provided on a line going through the center of beam incidence. Temperature is also plotted in the same diagram.

An interesting observation is that at the temperature with zero thermal expansion the deformation curve starts to invert itself, and emerges from its trough, forming a peaked structure. The most important parameter however is the slope caused by this deformation, as it affects the incidence angle of the rays. The derivative of the \textit{z} deformation with respect to \textit{y} is presented in Figure~\ref{fig:yslope}. For a reasonable maximum crystal temperature of 135~K the largest slope error is about 8~$\mu$rad. This 8~$\mu$rad is of the order of the Si~333 rocking curve widths in Figure~\ref{fig:333USG} and therefore significant changes in throughput can be expected. It is crucial then to model the deformation accurately and include it in ray tracing simulations.

\subsection{Artificial Gaussian Source}\label{gaussian}
\subsubsection{Si~111 with Two Dimensional Strain Field}\label{111simulation}
Earlier, it has been concluded that a uniform strain depth gradient in Si~111 has no considerable effect. However, yet to be investigated is strain constant in depth though varying across the crystal surface. This has the effect of shifting the rocking curve an amount proportional to the strain found at that point in the crystal. When the bandwidth of the monochromator is much smaller than the incident beam this effect can be essentially nullified by adjusting the angle of the second crystal. However, when the strain values deviate significantly across the crystal surface much flux is potentially lost.


Scans over the energies 6 to 35~keV were performed in MASH for Si~111 with and without strain and surface deformation included. Figure~\ref{fig:111flux} charts the results of double crystal reflectivity. In both cases the strained crystal curve follows closely that of the unstrained and at no energy does the relative difference between them reach above 3\%.

Figure~\ref{fig:111monobw} shows the monochromator bandwidths (standard deviation) as a function of set photon energy with or without strain and deformation. They are nearly the same, this time with a deviation below 4\%.

The conclusion is that the effects of strain in this model are negligible. However, it is still worth investigating Si~111 monochromators in more specific simulations with, e.g., insertion devices with collimated beams and beamline optics.



\subsubsection{Si~333 with Two Dimensional Strain Field}\label{parameterscans}

The simulations described in Section~\ref{111simulation} were repeated for Si~333. Unlike for Si~111 the results with and without strain were drastically dissimilar. In Figure~\ref{fig:333flux} discrepancy ranges from 60\% at 6~keV decreasing to below 50\% near 35~keV. The deviation in bandwidth, shown in Figure~\ref{fig:333monobw}, increases from near 0\% at 7~eV to about 25\% at 35~keV.


\subsubsection{Power}
Double Si~333 throughput as a function of photon energy is plotted in Figure~\ref{fig:333strainpower} for source powers 50, 100, 150 and 200 Watts. The top 4 lines are results for a Si~333 crystal without considering strain or deformation while the bottom four consider both. The flux is normalized with respect to the source flux of the 50~W case.

The results clearly show that increases in source power reduce the percentage of the reflected photon flux. This is due to more dramatic strains and deformations at higher power.

\subsubsection{Crystal Distance from Source}
Figure~\ref{fig:333straindistance} shows throughput for varying distances of the first crystal from the source. Throughput increases with increasing distance. This is due to the decrease of power per unit area and resulting reduction in strain and deformation. The throughput for 80~m distance from the source shows a small drop in flux at higher energies. This is due to the small Bragg angle at these energies and the beam footprint extending outside the crystal boundaries.

\subsubsection{Source Size}
In Figure~\ref{fig:333strainsourcesize} one sees that source size has no significant effect on flux within the range of the scan. This is because at 20~m the radiation cone has already spread wide enough to hide the micron level changes in the size of the source.

\subsubsection{Beam Divergence}
By increasing beam divergence the beam more quickly expands over a given path length. To take an isolated look at the effects of divergence the power density was held constant by adjusting the source distance for each divergence value. It appears from Figure~\ref{fig:333straindivergence} that divergence itself does not have an effect on throughput over the conditions of the simulation.

\subsection{Undulator and Wiggler Source}\label{undulatorwiggler}

\subsubsection{Undulator}
Comparison of total photon flux between perfect Si~111 crystals and strained ones is seen in Figure~\ref{fig:ivu111flux}. Reduction in flux is negligible especially at higher energies.

The case of Si~333 tells a different story: attenuation due to strain is significant. One may wonder if tweaking the angle of the second crystal can recover some of the lost flux. The second crystal angle was scanned within the range of $\pm$ 0.0436~mrad. The consequence of angle adjustment is a shifting rocking curve and change in bandwidth and total flux. Figure~\ref{fig:22kevangle} plots these effects for a monochromator setting of 22~keV. Figure~\ref{fig:maxangleflux} shows a chart of flux for the cases without strain, with strain but without second crystal optimization, and finally with strain and optimization. One can see that with adjustments to the second crystal some flux can be recovered but not completely.

\subsubsection{Wiggler and the Balder Beamline Optics}

\subsection{ForMAX and Fluid Cooled Monochromator}\label{formax}
\subsection{BioMAX Simulation}\label{biomax}
\subsection{NanoMAX Simulation}\label{nanomax}

\section{Outlook}

     % Appendices appear after the main body of the text. They are prefixed by
     % a single \appendix declaration, and are then structured just like the
     % body text.

\appendix

\section{FEA Parameters}\label{feaparameters}

This appendix tabulates parameter values for materials in FEA simulations. Values for silicon, copper, indium, and liquid nitrogen are found in Tables~\ref{siliconFEA},~\ref{copperFEA},~\ref{indiumFEA},~\ref{nitrogenFEA} respectively.
\vspace{1.5cm}



     %-------------------------------------------------------------------------
     % The back matter of the paper - acknowledgements and references
     %-------------------------------------------------------------------------

     % Acknowledgements come after the appendices

\ack{Acknowledgements}


Muchas gracias a todos

     % References are at the end of the document, between \begin{references}
     % and \end{references} tags. Each reference is in a~\reference entry.

% \begin{references}
%~\reference{Author, A. \& Author, B. (1984). \emph{Journal} \textbf{Vol}, 
% first page--last page.}
% \end{references}


%% Note added by Overleaf: If using bibtex, remove the "references" environment above, and uncomment the following lines.
\bibliographystyle{iucr}
\referencelist{iucr}

     %-------------------------------------------------------------------------
     % TABLES AND FIGURES SHOULD BE INSERTED AFTER THE MAIN BODY OF THE TEXT
     %-------------------------------------------------------------------------

     % Simple tables should use the tabular environment according to this
     % model

%\begin{table}
%\caption{Caption to table}
%\begin{tabular}{llcr}      % Alignment for each cell: l=left, c=center, r=right
% HEADING    & FOR        & EACH       & COLUMN     \\
%\hline
% entry      & entry      & entry      & entry      \\
% entry      & entry      & entry      & entry      \\
% entry      & entry      & entry      & entry      \\
%\end{tabular}
%\end{table}

     % Postscript figures can be included with multiple figure blocks

%\begin{figure}
%\caption{Caption describing figure.}
%\includegraphics[width = 8.85cm]{fig1}
%\end{figure}

\begin{figure}
\caption{Meshed ``bare silicon" COMSOL model. Close up of finely meshed area near beam incidence is shown.}
\includegraphics[width = 8.85cm]{images/bare_silicon.png}
\label{fig:bare_silicon}
\end{figure}

\begin{figure}
\caption{Unmeshed COMSOL model of the NanoMAX monochromator. \textbf{(a)} finely meshed beam incidence area, \textbf{(b)} silicon crystal body, \textbf{(c)} copper cooling blocks, \textbf{(d)} nitrogen filled copper cooling pipes brazed to cooling block bodies}
\includegraphics[width = 8.85cm]{images/nanomaxcomsol.png}
\label{fig:nanomaxcomsol}
\end{figure}


\begin{figure}
\caption{Diagram of beamline setup for Gaussian source simulations. From left to right are the X-ray source, the first crystal and the second. Absolute and relative distances are given in cm.}
\includegraphics[width = 8.85cm]{images/gaussian_beamline.png}
\label{fig:dcmtracing}
\end{figure}



\begin{table}\label{gaussian_table}
\caption{Gaussian beam parameters}
\begin{tabular}{@{}llll@{}}
Parameter       & Value         & Units     & Description                                          \\
\hline
x               & 0.005         & cm        & Gaussian $\sigma$ of horizontal intensity            \\
z               & 0.0005        & cm        & Gaussian $\sigma$ of vertical intensity              \\ 
y               & 0.0005        & cm        & Gaussian $\sigma$ of longitudinal intensity          \\
x'              & 50            & mrad      & Gaussian $\sigma$ of horizontal angular intensity    \\
z'              & 50            & mrad      & Gaussian $\sigma$ of horizontal angular intensity    \\

\end{tabular}
\end{table}

\begin{table}\label{ivubiomax}
\caption{Undulator Parameters}
\begin{tabular}{@{}llll@{}}
Parameter       & Value         & Units     & Description                           \\
\hline
$\lambda$       & 1.8           & cm        & magnetic period length                \\
n               & 111           & 1         & number of magnetic periods            \\ 
$l$             & 1             & m         & magnetic length                       \\
$K$             & 2             & 1         & max K value                           \\
$\sigma_x$      & 40            & $\mu$m    & electron beam horizontal rms size     \\
$\sigma_z$      & 2             & $\mu$m    & electron beam vertical rms size       \\
$\epsilon_x$    & 3.3e-08       & cm$\cdot$rad    & electron beam horizontal emittance    \\
$\epsilon_z$    & 8e-10         & cm$\cdot$rad    & electron beam vertical emittance      \\
I               & 500           & mA        & electron beam current                 \\
E               & 3             & GeV       & electron energy                       \\
\end{tabular}
\end{table}

\begin{table}\label{ivwbalder}
\caption{Wiggler Parameters}
\begin{tabular}{@{}llll@{}}
Parameter       & Value         & Units     & Description                           \\
\hline
$\lambda$       & 5             & cm        & magnetic period length                \\
n               & 40            & 1         & number of magnetic periods            \\ 
$l$             & 1             & m         & magnetic length                       \\
$K$             & 9             & 1         & max K value                           \\
$\sigma_x$      & 50            & $\mu$m    & electron beam horizontal rms size     \\
$\sigma_z$      & 2             & $\mu$m    & electron beam vertical rms size       \\
$\epsilon_x$    & 3.3e-08       & cm$\cdot$rad    & electron beam horizontal emittance    \\
$\epsilon_z$    & 8e-10         & cm$\cdot$rad    & electron beam vertical emittance      \\
I               & 500           & mA        & electron beam current                 \\
E               & 3             & GeV       & electron energy                       \\
\end{tabular}
\end{table}


\begin{figure}
\caption{Reflectively of 10~keV \textbf{(a)}, \textbf{(b)}, and 20~keV \textbf{(c)}, \textbf{(d)}, photons over deviations from Bragg angle in Si~111 with uniform strain gradient values and max temperatures of \textbf{(a)} 1.84e-02, 133~K, \textbf{(b)}  1.27e-01, 168~K, \textbf{(c)} 1.84e-02, 133~K, and \textbf{(d)}  1.27e-01, 168~K}
\includegraphics[width =\textwidth]{images/111USG.png}
\label{fig:111USG}
\end{figure}

\begin{figure}
\caption{Reflectively of 10~keV \textbf{(a)}, \textbf{(b)}, and 20~keV \textbf{(c)}, \textbf{(d)}, photons over deviations from Bragg angle in Si~333 with uniform strain gradient values and max temperatures of \textbf{(a)} 1.84e-02, 133~K, \textbf{(b)}  1.27e-01, 168~K, \textbf{(c)} 1.84e-02, 133~K, and \textbf{(d)}  1.27e-01, 168~K}
\includegraphics[width = \textwidth]{images/333USG.png}
\label{fig:333USG}
\end{figure}



\begin{figure}

\caption{Charts of deformation and temperature along the center of the surface of silicon crystal in the meridional direction. Temperature is given by the green dashed line and deformation by the blue solid line. Max temperatures in the crystal are \textbf{(a)} 133~K, \textbf{(b)} 144~K, \textbf{(c)} 156~K, \textbf{(d)} 168~K, \textbf{(e)} 180~K, and \textbf{f)} 192~K. Charts generated from simulations performed in COMSOL.}
\includegraphics[width = \textwidth]{images/deformation.png}

\label{fig:ydeformation}
\end{figure}

\begin{figure}
\caption{Slope surface deformation of 1st crystal in the meridional direction. Legend gives associated maximum temperature in the crystal.}
\includegraphics[width = 8.85cm]{images/slope.png}
\label{fig:yslope}
\end{figure}

\begin{figure}
\caption{Double crystal photon flux in Si~111 for strained and unstrained cases using a Gaussian source}
\includegraphics[width = 8.85cm]{images/111flux.png}
\label{fig:111flux}
\end{figure}

\begin{figure}
\caption{Double crystal monochromator bandwidth in Si~111 for strained and unstrained cases using a Gaussian source}
\includegraphics[width = 8.85cm]{images/111monobw.png}
\label{fig:111monobw}
\end{figure}

\begin{figure}
\caption{Double crystal photon flux in Si~333 for strained and unstrained cases using a Gaussian source}
\includegraphics[width = 8.85cm]{images/333flux.png}
\label{fig:333flux}
\end{figure}

\begin{figure}
\caption{Double crystal monochromator bandwidth in Si~333 for strained and unstrained cases using a Gaussian source}
\includegraphics[width = 8.85cm]{images/333monobw.png}
\label{fig:333monobw}
\end{figure}

\begin{figure}
\caption{Double crystal photon flux in Si~333 for strained and unstrained cases using a Gaussian source. Source power is varied and stated by the legend in units of Watt. Photon flux is normalized to the 50~W case. The upper four lines are without strain while the bottom four includes it.}
\includegraphics[width = 8.85cm]{images/333strainpower.png}
\label{fig:333strainpower}
\end{figure}


\begin{figure}
\caption{Double crystal photon flux in Si~333 for strained and unstrained cases using a Gaussian source. Source distance is varied and stated by the legend in units of meter. The upper four lines are without strain while the bottom four includes it.}
\includegraphics[width = 8.85cm]{images/333straindistance.png}
\label{fig:333straindistance}
\end{figure}

\begin{figure}
\caption{Double crystal photon flux in Si~333 for strained and unstrained cases using a Gaussian source. Source divergence is varied as (25,25), (50,50), (75,75), (100,100) units of mrad$^2$. Power per area on crystal surface is normalized to the 25$\times$25 mrad$^2$ case. The upper four lines are without strain while the bottom four includes it.}
\includegraphics[width = 8.85cm]{images/333straindivergence.png}
\label{fig:333straindivergence}
\end{figure}

\begin{figure}
\caption{Double crystal photon flux in Si~333 for strained and unstrained cases using a Gaussian source. Source size is varied as (0,0), (25,2.5), (50,5.0), (75,7.5), and (100,10) in units of micron$^2$. The upper four lines are without strain while the bottom four includes it.}
\includegraphics[width = 8.85cm]{images/333strainsize.png}
\label{fig:333strainsourcesize}
\end{figure}

\begin{figure}
\caption{Double crystal photon flux in Si~111 for strained and unstrained cases using an undulator source with parameters given in Table~\ref{ivubiomax}.}.
\includegraphics[width = 8.85cm]{images/ivu111flux.png}
\label{fig:ivu111flux}
\end{figure}

\begin{figure}
\caption{Double crystal photon flux in Si~333 for strained and unstrained cases using an undulator source with parameters given in Table~\ref{ivubiomax}.}
\includegraphics[width = 8.85cm]{images/ivu333flux.png}
\label{fig:ivu333flux}
\end{figure}

\begin{figure}
\caption{Double crystal photon flux in Si~333 at 22~keV for unstrained case as well as strained cases for second crystal tweaking angles stated by the legend in degrees. Undulator source used with parameters given in Table~\ref{ivubiomax}.}
\includegraphics[width = 8.85cm]{images/22kevangle.png}
\label{fig:22kevangle}
\end{figure}

\begin{figure}
\caption{Double crystal photon flux in Si~333 unstrained and strained cases as well as case for maximum optimization angle of second crystal. Undulator source used with parameters given in Table~\ref{ivubiomax}.}
\includegraphics[width = 8.85cm]{images/maxangleflux.png}
\label{fig:maxangleflux}
\end{figure}

\begin{table}

\caption{Silicon FEA Parameters}
\begin{tabular}{@{}llll@{}}
Parameter   & Value                   & Units                     & Description                        \\
\hline
$h$           & 0.3                     & W/cm$^2\cdot$K          & Heat transfer coefficient          \\
$T_{cool}$  & 77                      & K                         & Side cooling temperature           \\
$C_{11}$    & $1.6772\times 10^{11}$  & Pa                        & Stiffness tensor 11 component      \\
$C_{12}$    & $6.4980\times 10^{10}$  & Pa                        & Stiffness tensor 12 component      \\
$C_{44}$    & $8.0360\times 10^{10}$  & Pa                        & Stiffness tensor 44 component      \\
$C_p$       & 0.2                     & J/g$\cdot$K               & Heat capacity at constant pressure \\
$\rho$      &  2329                   & kg/m$^3$                  & Density                            \\
\end{tabular}
\label{siliconFEA}
\end{table}


\begin{table}

\caption{Copper FEA Parameters}
\begin{tabular}{@{}llll@{}}
Parameter    & Value                  & Units                      & Description                        \\
\hline
$C_p$        & 385                    & J/kg$\cdot$K               & Heat capacity at constant pressure \\
$\kappa$     & 400                    & W/m$\cdot$K                & Thermal conductivity               \\ 
$\rho$       & 8960                   & kg/m$^3$                   & Density                            \\
\end{tabular}
\label{copperFEA}
\end{table}


\begin{table}
\caption{Indium FEA Parameters}
\begin{tabular}{@{}llll@{}}
Parameter    & Value                  & Units                      & Description                        \\
\hline
$C_p$        & 233                    & J/kg$\cdot$K               & Heat capacity at constant pressure \\
$\kappa$     & 81.6                   & W/m$\cdot$K                & Thermal conductivity               \\
$\rho$       & 7290                   & kg/m$^3$                   & Density                            \\
\end{tabular}
\label{indiumFEA}
\end{table}


\begin{table}
\caption{Liquid Nitrogen FEA Parameters}
\begin{tabular}{@{}llll@{}}
Parameter    & Value                  & Units                       & Description                        \\
\hline
$\mu$        & $157.9\times 10^{-6}$  & Pa$\cdot$s                  & Dynamic viscosity                  \\ 
$\gamma$     & 1.47                   & 1				            & Ratio of specific heats            \\
$C_p$        & 2.04                   & kJ/kg$\cdot$K               & Heat capacity at constant pressure \\
$\kappa$     & 139.6                  & mW/m$\cdot$K                & Thermal conductivity               \\
$\rho$       & 0.807                  & g/ml                        & Density                            \\
\end{tabular}
\label{nitrogenFEA}
\end{table}

\end{document}                    % DO NOT DELETE THIS LINE
%%%%%%%%%%%%%%%%%%%%%%%%%%%%%%%%%%%%%%%%%%%%%%%%%%%%%%%%%%%%%%%%%%%%%%%%%%%%%%